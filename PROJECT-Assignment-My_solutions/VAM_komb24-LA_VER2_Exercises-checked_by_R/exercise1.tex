% Options for packages loaded elsewhere
\PassOptionsToPackage{unicode}{hyperref}
\PassOptionsToPackage{hyphens}{url}
%
\documentclass[
  10pt,
  a4paper,
]{article}
\usepackage{amsmath,amssymb}
\usepackage{iftex}
\ifPDFTeX
  \usepackage[T1]{fontenc}
  \usepackage[utf8]{inputenc}
  \usepackage{textcomp} % provide euro and other symbols
\else % if luatex or xetex
  \usepackage{unicode-math} % this also loads fontspec
  \defaultfontfeatures{Scale=MatchLowercase}
  \defaultfontfeatures[\rmfamily]{Ligatures=TeX,Scale=1}
\fi
\usepackage{lmodern}
\ifPDFTeX\else
  % xetex/luatex font selection
\fi
% Use upquote if available, for straight quotes in verbatim environments
\IfFileExists{upquote.sty}{\usepackage{upquote}}{}
\IfFileExists{microtype.sty}{% use microtype if available
  \usepackage[]{microtype}
  \UseMicrotypeSet[protrusion]{basicmath} % disable protrusion for tt fonts
}{}
\makeatletter
\@ifundefined{KOMAClassName}{% if non-KOMA class
  \IfFileExists{parskip.sty}{%
    \usepackage{parskip}
  }{% else
    \setlength{\parindent}{0pt}
    \setlength{\parskip}{6pt plus 2pt minus 1pt}}
}{% if KOMA class
  \KOMAoptions{parskip=half}}
\makeatother
\usepackage{xcolor}
\usepackage{color}
\usepackage{fancyvrb}
\newcommand{\VerbBar}{|}
\newcommand{\VERB}{\Verb[commandchars=\\\{\}]}
\DefineVerbatimEnvironment{Highlighting}{Verbatim}{commandchars=\\\{\}}
% Add ',fontsize=\small' for more characters per line
\usepackage{framed}
\definecolor{shadecolor}{RGB}{248,248,248}
\newenvironment{Shaded}{\begin{snugshade}}{\end{snugshade}}
\newcommand{\AlertTok}[1]{\textcolor[rgb]{0.94,0.16,0.16}{#1}}
\newcommand{\AnnotationTok}[1]{\textcolor[rgb]{0.56,0.35,0.01}{\textbf{\textit{#1}}}}
\newcommand{\AttributeTok}[1]{\textcolor[rgb]{0.13,0.29,0.53}{#1}}
\newcommand{\BaseNTok}[1]{\textcolor[rgb]{0.00,0.00,0.81}{#1}}
\newcommand{\BuiltInTok}[1]{#1}
\newcommand{\CharTok}[1]{\textcolor[rgb]{0.31,0.60,0.02}{#1}}
\newcommand{\CommentTok}[1]{\textcolor[rgb]{0.56,0.35,0.01}{\textit{#1}}}
\newcommand{\CommentVarTok}[1]{\textcolor[rgb]{0.56,0.35,0.01}{\textbf{\textit{#1}}}}
\newcommand{\ConstantTok}[1]{\textcolor[rgb]{0.56,0.35,0.01}{#1}}
\newcommand{\ControlFlowTok}[1]{\textcolor[rgb]{0.13,0.29,0.53}{\textbf{#1}}}
\newcommand{\DataTypeTok}[1]{\textcolor[rgb]{0.13,0.29,0.53}{#1}}
\newcommand{\DecValTok}[1]{\textcolor[rgb]{0.00,0.00,0.81}{#1}}
\newcommand{\DocumentationTok}[1]{\textcolor[rgb]{0.56,0.35,0.01}{\textbf{\textit{#1}}}}
\newcommand{\ErrorTok}[1]{\textcolor[rgb]{0.64,0.00,0.00}{\textbf{#1}}}
\newcommand{\ExtensionTok}[1]{#1}
\newcommand{\FloatTok}[1]{\textcolor[rgb]{0.00,0.00,0.81}{#1}}
\newcommand{\FunctionTok}[1]{\textcolor[rgb]{0.13,0.29,0.53}{\textbf{#1}}}
\newcommand{\ImportTok}[1]{#1}
\newcommand{\InformationTok}[1]{\textcolor[rgb]{0.56,0.35,0.01}{\textbf{\textit{#1}}}}
\newcommand{\KeywordTok}[1]{\textcolor[rgb]{0.13,0.29,0.53}{\textbf{#1}}}
\newcommand{\NormalTok}[1]{#1}
\newcommand{\OperatorTok}[1]{\textcolor[rgb]{0.81,0.36,0.00}{\textbf{#1}}}
\newcommand{\OtherTok}[1]{\textcolor[rgb]{0.56,0.35,0.01}{#1}}
\newcommand{\PreprocessorTok}[1]{\textcolor[rgb]{0.56,0.35,0.01}{\textit{#1}}}
\newcommand{\RegionMarkerTok}[1]{#1}
\newcommand{\SpecialCharTok}[1]{\textcolor[rgb]{0.81,0.36,0.00}{\textbf{#1}}}
\newcommand{\SpecialStringTok}[1]{\textcolor[rgb]{0.31,0.60,0.02}{#1}}
\newcommand{\StringTok}[1]{\textcolor[rgb]{0.31,0.60,0.02}{#1}}
\newcommand{\VariableTok}[1]{\textcolor[rgb]{0.00,0.00,0.00}{#1}}
\newcommand{\VerbatimStringTok}[1]{\textcolor[rgb]{0.31,0.60,0.02}{#1}}
\newcommand{\WarningTok}[1]{\textcolor[rgb]{0.56,0.35,0.01}{\textbf{\textit{#1}}}}
\usepackage{graphicx}
\makeatletter
\def\maxwidth{\ifdim\Gin@nat@width>\linewidth\linewidth\else\Gin@nat@width\fi}
\def\maxheight{\ifdim\Gin@nat@height>\textheight\textheight\else\Gin@nat@height\fi}
\makeatother
% Scale images if necessary, so that they will not overflow the page
% margins by default, and it is still possible to overwrite the defaults
% using explicit options in \includegraphics[width, height, ...]{}
\setkeys{Gin}{width=\maxwidth,height=\maxheight,keepaspectratio}
% Set default figure placement to htbp
\makeatletter
\def\fps@figure{htbp}
\makeatother
\setlength{\emergencystretch}{3em} % prevent overfull lines
\providecommand{\tightlist}{%
  \setlength{\itemsep}{0pt}\setlength{\parskip}{0pt}}
\setcounter{secnumdepth}{-\maxdimen} % remove section numbering
\usepackage[margin=0.75in]{geometry}
\usepackage{fancyhdr}
\let\proof\relax
\let\endproof\relax
\usepackage{amsmath}
\pagestyle{fancy}
\fancyhead{}
\fancyfoot{}
\lhead{LA1 Project}
\rhead{Exercise 1 (Version 2)}
\lfoot{January, 2025}\rfoot{Pg. \thepage}

\ifLuaTeX
  \usepackage{selnolig}  % disable illegal ligatures
\fi
\IfFileExists{bookmark.sty}{\usepackage{bookmark}}{\usepackage{hyperref}}
\IfFileExists{xurl.sty}{\usepackage{xurl}}{} % add URL line breaks if available
\urlstyle{same}
% Make links footnotes instead of hotlinks:
\DeclareRobustCommand{\href}[2]{#2\footnote{\url{#1}}}
\hypersetup{
  pdftitle={LA1 - Version 2 - Exercise 1},
  pdfauthor={Ondrej Salamon},
  hidelinks,
  pdfcreator={LaTeX via pandoc}}

\title{LA1 - Version 2 - Exercise 1}
\author{Ondrej Salamon}
\date{}

\begin{document}
\maketitle

\begin{Shaded}
\begin{Highlighting}[]
\CommentTok{\# initial setup}
\FunctionTok{options}\NormalTok{(}\AttributeTok{scipen =} \DecValTok{999}\NormalTok{)}
\FunctionTok{options}\NormalTok{(}\AttributeTok{tinytex.verbose =} \ConstantTok{TRUE}\NormalTok{)}
\FunctionTok{library}\NormalTok{(matlib)}
\FunctionTok{library}\NormalTok{(knitr)}
\FunctionTok{library}\NormalTok{(rmarkdown)}
\FunctionTok{library}\NormalTok{(quarto)}
\FunctionTok{library}\NormalTok{(tinytex)}
\FunctionTok{library}\NormalTok{(pandoc)}
\NormalTok{knitr}\SpecialCharTok{::}\NormalTok{opts\_chunk}\SpecialCharTok{$}\FunctionTok{set}\NormalTok{(}\AttributeTok{echo=}\ConstantTok{TRUE}\NormalTok{, }\AttributeTok{message=}\ConstantTok{FALSE}\NormalTok{, }\AttributeTok{warning=}\ConstantTok{FALSE}\NormalTok{, }\AttributeTok{fig.width=}\DecValTok{6}\NormalTok{, }\AttributeTok{fig.height=}\DecValTok{6}\NormalTok{)}
\end{Highlighting}
\end{Shaded}

\hypertarget{find-all-solution-of-systems-of-linear-equations-of-the-form-axb-where}{%
\subsection{\texorpdfstring{Find all solution of systems of linear
equations of the form \textbf{Ax=b},
where:}{Find all solution of systems of linear equations of the form Ax=b, where:}}\label{find-all-solution-of-systems-of-linear-equations-of-the-form-axb-where}}

\begin{Shaded}
\begin{Highlighting}[]
\CommentTok{\# enter variable (matrix) A in R matlib pkg}
\NormalTok{A }\OtherTok{\textless{}{-}} \FunctionTok{matrix}\NormalTok{(}
        \FunctionTok{c}\NormalTok{(}\SpecialCharTok{{-}}\DecValTok{2}\NormalTok{, }\DecValTok{1}\NormalTok{, }\DecValTok{0}\NormalTok{,  }\DecValTok{2}\NormalTok{,}
           \DecValTok{3}\NormalTok{,}\SpecialCharTok{{-}}\DecValTok{2}\NormalTok{, }\DecValTok{0}\NormalTok{, }\SpecialCharTok{{-}}\DecValTok{5}\NormalTok{,}
          \SpecialCharTok{{-}}\DecValTok{2}\NormalTok{, }\DecValTok{3}\NormalTok{, }\DecValTok{2}\NormalTok{, }\DecValTok{12}\NormalTok{,}
           \DecValTok{2}\NormalTok{,}\SpecialCharTok{{-}}\DecValTok{2}\NormalTok{,}\SpecialCharTok{{-}}\DecValTok{1}\NormalTok{, }\SpecialCharTok{{-}}\DecValTok{7}\NormalTok{), }\DecValTok{4}\NormalTok{, }\DecValTok{4}\NormalTok{, }\AttributeTok{byrow=}\ConstantTok{TRUE}\NormalTok{)}
\end{Highlighting}
\end{Shaded}

\begin{Shaded}
\begin{Highlighting}[]
\CommentTok{\# enter variable with right hand side vertical vector}
\NormalTok{b }\OtherTok{\textless{}{-}} \FunctionTok{c}\NormalTok{(}\SpecialCharTok{{-}}\DecValTok{3}\NormalTok{, }\DecValTok{5}\NormalTok{, }\SpecialCharTok{{-}}\DecValTok{3}\NormalTok{, }\DecValTok{3}\NormalTok{)}
\end{Highlighting}
\end{Shaded}

\hypertarget{code-the-equation-in-r-matlib-commented-out}{%
\subsubsection{Code the equation in R matlib (commented
out):}\label{code-the-equation-in-r-matlib-commented-out}}

\begin{verbatim}
Eqn(
    "\\mathbf{A} =",
     latexMatrix(A, matrix="bmatrix"),
     latexMatrix("x", nrow = 4, ncol=1),
     Eqn_hspace(mid='='), 
     latexMatrix(matrix(b, ncol = 1))
)
\end{verbatim}

\hypertarget{show-equation-rs-latex-output}{%
\subsubsection{Show equation (R's Latex
output):}\label{show-equation-rs-latex-output}}

\begin{equation*}
\mathbf{A} =\begin{bmatrix} 
-2 &  1 &  0 &  2 \\ 
 3 & -2 &  0 & -5 \\ 
-2 &  3 &  2 & 12 \\ 
 2 & -2 & -1 & -7 \\ 
\end{bmatrix}
\begin{pmatrix} 
  x_{1} \\ 
  x_{2} \\ 
  x_{3} \\ 
  x_{4} \\ 
\end{pmatrix}
\quad=\quad\begin{pmatrix} 
-3 \\ 
 5 \\ 
-3 \\ 
 3 \\ 
\end{pmatrix}
\end{equation*}

\hypertarget{code-equation-solving}{%
\subsubsection{Code equation solving:}\label{code-equation-solving}}

\begin{verbatim}
Solve(A, b, fractions = TRUE)
\end{verbatim}

\hypertarget{show-the-final-result-of-the-equation-not-latex-plain}{%
\subsubsection{Show the final result of the equation (not Latex,
plain):}\label{show-the-final-result-of-the-equation-not-latex-plain}}

\begin{verbatim}
x1       + x4  =   1 
  x2   + 4*x4  =  -1 
    x3   + x4  =   1 
            0  =   0 
\end{verbatim}

\hypertarget{show-whole-verbose-solution-ie.-each-row-operation-created-by-command-solve}{%
\subsubsection{\texorpdfstring{Show whole verbose solution, ie. each row
operation, created by command
\texttt{Solve}:}{Show whole verbose solution, ie. each row operation, created by command Solve:}}\label{show-whole-verbose-solution-ie.-each-row-operation-created-by-command-solve}}

\begin{Shaded}
\begin{Highlighting}[]
\CommentTok{\# thx to \textasciigrave{}verbose = TRUE\textasciigrave{} it outputs solution step by}
\CommentTok{\# step with Gaussian elementary row operations}
\FunctionTok{Solve}\NormalTok{(A, b, }\AttributeTok{verbose =} \ConstantTok{TRUE}\NormalTok{, }\AttributeTok{fractions =} \ConstantTok{TRUE}\NormalTok{)}
\end{Highlighting}
\end{Shaded}

\begin{verbatim}
## 
## Initial matrix:
\end{verbatim}

\begin{verbatim}
##      [,1] [,2] [,3] [,4] [,5]
## [1,] -2    1    0    2   -3  
## [2,]  3   -2    0   -5    5  
## [3,] -2    3    2   12   -3  
## [4,]  2   -2   -1   -7    3  
## 
## row: 1 
## 
##  exchange rows 1 and 2
\end{verbatim}

\begin{verbatim}
##      [,1] [,2] [,3] [,4] [,5]
## [1,]  3   -2    0   -5    5  
## [2,] -2    1    0    2   -3  
## [3,] -2    3    2   12   -3  
## [4,]  2   -2   -1   -7    3  
## 
##  multiply row 1 by 1/3
\end{verbatim}

\begin{verbatim}
##      [,1] [,2] [,3] [,4] [,5]
## [1,]    1 -2/3    0 -5/3  5/3
## [2,]   -2    1    0    2   -3
## [3,]   -2    3    2   12   -3
## [4,]    2   -2   -1   -7    3
## 
##  multiply row 1 by 2 and add to row 2
\end{verbatim}

\begin{verbatim}
##      [,1] [,2] [,3] [,4] [,5]
## [1,]    1 -2/3    0 -5/3  5/3
## [2,]    0 -1/3    0 -4/3  1/3
## [3,]   -2    3    2   12   -3
## [4,]    2   -2   -1   -7    3
## 
##  multiply row 1 by 2 and add to row 3
\end{verbatim}

\begin{verbatim}
##      [,1] [,2] [,3] [,4] [,5]
## [1,]    1 -2/3    0 -5/3  5/3
## [2,]    0 -1/3    0 -4/3  1/3
## [3,]    0  5/3    2 26/3  1/3
## [4,]    2   -2   -1   -7    3
## 
##  multiply row 1 by 2 and subtract from row 4
\end{verbatim}

\begin{verbatim}
##      [,1]  [,2]  [,3]  [,4]  [,5] 
## [1,]     1  -2/3     0  -5/3   5/3
## [2,]     0  -1/3     0  -4/3   1/3
## [3,]     0   5/3     2  26/3   1/3
## [4,]     0  -2/3    -1 -11/3  -1/3
## 
## row: 2 
## 
##  exchange rows 2 and 3
\end{verbatim}

\begin{verbatim}
##      [,1]  [,2]  [,3]  [,4]  [,5] 
## [1,]     1  -2/3     0  -5/3   5/3
## [2,]     0   5/3     2  26/3   1/3
## [3,]     0  -1/3     0  -4/3   1/3
## [4,]     0  -2/3    -1 -11/3  -1/3
## 
##  multiply row 2 by 3/5
\end{verbatim}

\begin{verbatim}
##      [,1]  [,2]  [,3]  [,4]  [,5] 
## [1,]     1  -2/3     0  -5/3   5/3
## [2,]     0     1   6/5  26/5   1/5
## [3,]     0  -1/3     0  -4/3   1/3
## [4,]     0  -2/3    -1 -11/3  -1/3
## 
##  multiply row 2 by 2/3 and add to row 1
\end{verbatim}

\begin{verbatim}
##      [,1]  [,2]  [,3]  [,4]  [,5] 
## [1,]     1     0   4/5   9/5   9/5
## [2,]     0     1   6/5  26/5   1/5
## [3,]     0  -1/3     0  -4/3   1/3
## [4,]     0  -2/3    -1 -11/3  -1/3
## 
##  multiply row 2 by 1/3 and add to row 3
\end{verbatim}

\begin{verbatim}
##      [,1]  [,2]  [,3]  [,4]  [,5] 
## [1,]     1     0   4/5   9/5   9/5
## [2,]     0     1   6/5  26/5   1/5
## [3,]     0     0   2/5   2/5   2/5
## [4,]     0  -2/3    -1 -11/3  -1/3
## 
##  multiply row 2 by 2/3 and add to row 4
\end{verbatim}

\begin{verbatim}
##      [,1] [,2] [,3] [,4] [,5]
## [1,]    1    0  4/5  9/5  9/5
## [2,]    0    1  6/5 26/5  1/5
## [3,]    0    0  2/5  2/5  2/5
## [4,]    0    0 -1/5 -1/5 -1/5
## 
## row: 3 
## 
##  multiply row 3 by 5/2
\end{verbatim}

\begin{verbatim}
##      [,1] [,2] [,3] [,4] [,5]
## [1,]    1    0  4/5  9/5  9/5
## [2,]    0    1  6/5 26/5  1/5
## [3,]    0    0    1    1    1
## [4,]    0    0 -1/5 -1/5 -1/5
## 
##  multiply row 3 by 4/5 and subtract from row 1
\end{verbatim}

\begin{verbatim}
##      [,1] [,2] [,3] [,4] [,5]
## [1,]    1    0    0    1    1
## [2,]    0    1  6/5 26/5  1/5
## [3,]    0    0    1    1    1
## [4,]    0    0 -1/5 -1/5 -1/5
## 
##  multiply row 3 by 6/5 and subtract from row 2
\end{verbatim}

\begin{verbatim}
##      [,1] [,2] [,3] [,4] [,5]
## [1,]    1    0    0    1    1
## [2,]    0    1    0    4   -1
## [3,]    0    0    1    1    1
## [4,]    0    0 -1/5 -1/5 -1/5
## 
##  multiply row 3 by 1/5 and add to row 4
\end{verbatim}

\begin{verbatim}
##      [,1] [,2] [,3] [,4] [,5]
## [1,]  1    0    0    1    1  
## [2,]  0    1    0    4   -1  
## [3,]  0    0    1    1    1  
## [4,]  0    0    0    0    0  
## 
## row: 4 
## x1       + x4  =   1 
##   x2   + 4*x4  =  -1 
##     x3   + x4  =   1 
##             0  =   0
\end{verbatim}

\end{document}
