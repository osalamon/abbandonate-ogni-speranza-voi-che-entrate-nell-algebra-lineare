% Options for packages loaded elsewhere
\PassOptionsToPackage{unicode}{hyperref}
\PassOptionsToPackage{hyphens}{url}
%
\documentclass[
]{article}
\usepackage{amsmath,amssymb}
\usepackage{iftex}
\ifPDFTeX
  \usepackage[T1]{fontenc}
  \usepackage[utf8]{inputenc}
  \usepackage{textcomp} % provide euro and other symbols
\else % if luatex or xetex
  \usepackage{unicode-math} % this also loads fontspec
  \defaultfontfeatures{Scale=MatchLowercase}
  \defaultfontfeatures[\rmfamily]{Ligatures=TeX,Scale=1}
\fi
\usepackage{lmodern}
\ifPDFTeX\else
  % xetex/luatex font selection
\fi
% Use upquote if available, for straight quotes in verbatim environments
\IfFileExists{upquote.sty}{\usepackage{upquote}}{}
\IfFileExists{microtype.sty}{% use microtype if available
  \usepackage[]{microtype}
  \UseMicrotypeSet[protrusion]{basicmath} % disable protrusion for tt fonts
}{}
\makeatletter
\@ifundefined{KOMAClassName}{% if non-KOMA class
  \IfFileExists{parskip.sty}{%
    \usepackage{parskip}
  }{% else
    \setlength{\parindent}{0pt}
    \setlength{\parskip}{6pt plus 2pt minus 1pt}}
}{% if KOMA class
  \KOMAoptions{parskip=half}}
\makeatother
\usepackage{xcolor}
\usepackage[margin=1in]{geometry}
\usepackage{color}
\usepackage{fancyvrb}
\newcommand{\VerbBar}{|}
\newcommand{\VERB}{\Verb[commandchars=\\\{\}]}
\DefineVerbatimEnvironment{Highlighting}{Verbatim}{commandchars=\\\{\}}
% Add ',fontsize=\small' for more characters per line
\usepackage{framed}
\definecolor{shadecolor}{RGB}{248,248,248}
\newenvironment{Shaded}{\begin{snugshade}}{\end{snugshade}}
\newcommand{\AlertTok}[1]{\textcolor[rgb]{0.94,0.16,0.16}{#1}}
\newcommand{\AnnotationTok}[1]{\textcolor[rgb]{0.56,0.35,0.01}{\textbf{\textit{#1}}}}
\newcommand{\AttributeTok}[1]{\textcolor[rgb]{0.13,0.29,0.53}{#1}}
\newcommand{\BaseNTok}[1]{\textcolor[rgb]{0.00,0.00,0.81}{#1}}
\newcommand{\BuiltInTok}[1]{#1}
\newcommand{\CharTok}[1]{\textcolor[rgb]{0.31,0.60,0.02}{#1}}
\newcommand{\CommentTok}[1]{\textcolor[rgb]{0.56,0.35,0.01}{\textit{#1}}}
\newcommand{\CommentVarTok}[1]{\textcolor[rgb]{0.56,0.35,0.01}{\textbf{\textit{#1}}}}
\newcommand{\ConstantTok}[1]{\textcolor[rgb]{0.56,0.35,0.01}{#1}}
\newcommand{\ControlFlowTok}[1]{\textcolor[rgb]{0.13,0.29,0.53}{\textbf{#1}}}
\newcommand{\DataTypeTok}[1]{\textcolor[rgb]{0.13,0.29,0.53}{#1}}
\newcommand{\DecValTok}[1]{\textcolor[rgb]{0.00,0.00,0.81}{#1}}
\newcommand{\DocumentationTok}[1]{\textcolor[rgb]{0.56,0.35,0.01}{\textbf{\textit{#1}}}}
\newcommand{\ErrorTok}[1]{\textcolor[rgb]{0.64,0.00,0.00}{\textbf{#1}}}
\newcommand{\ExtensionTok}[1]{#1}
\newcommand{\FloatTok}[1]{\textcolor[rgb]{0.00,0.00,0.81}{#1}}
\newcommand{\FunctionTok}[1]{\textcolor[rgb]{0.13,0.29,0.53}{\textbf{#1}}}
\newcommand{\ImportTok}[1]{#1}
\newcommand{\InformationTok}[1]{\textcolor[rgb]{0.56,0.35,0.01}{\textbf{\textit{#1}}}}
\newcommand{\KeywordTok}[1]{\textcolor[rgb]{0.13,0.29,0.53}{\textbf{#1}}}
\newcommand{\NormalTok}[1]{#1}
\newcommand{\OperatorTok}[1]{\textcolor[rgb]{0.81,0.36,0.00}{\textbf{#1}}}
\newcommand{\OtherTok}[1]{\textcolor[rgb]{0.56,0.35,0.01}{#1}}
\newcommand{\PreprocessorTok}[1]{\textcolor[rgb]{0.56,0.35,0.01}{\textit{#1}}}
\newcommand{\RegionMarkerTok}[1]{#1}
\newcommand{\SpecialCharTok}[1]{\textcolor[rgb]{0.81,0.36,0.00}{\textbf{#1}}}
\newcommand{\SpecialStringTok}[1]{\textcolor[rgb]{0.31,0.60,0.02}{#1}}
\newcommand{\StringTok}[1]{\textcolor[rgb]{0.31,0.60,0.02}{#1}}
\newcommand{\VariableTok}[1]{\textcolor[rgb]{0.00,0.00,0.00}{#1}}
\newcommand{\VerbatimStringTok}[1]{\textcolor[rgb]{0.31,0.60,0.02}{#1}}
\newcommand{\WarningTok}[1]{\textcolor[rgb]{0.56,0.35,0.01}{\textbf{\textit{#1}}}}
\usepackage{graphicx}
\makeatletter
\def\maxwidth{\ifdim\Gin@nat@width>\linewidth\linewidth\else\Gin@nat@width\fi}
\def\maxheight{\ifdim\Gin@nat@height>\textheight\textheight\else\Gin@nat@height\fi}
\makeatother
% Scale images if necessary, so that they will not overflow the page
% margins by default, and it is still possible to overwrite the defaults
% using explicit options in \includegraphics[width, height, ...]{}
\setkeys{Gin}{width=\maxwidth,height=\maxheight,keepaspectratio}
% Set default figure placement to htbp
\makeatletter
\def\fps@figure{htbp}
\makeatother
\setlength{\emergencystretch}{3em} % prevent overfull lines
\providecommand{\tightlist}{%
  \setlength{\itemsep}{0pt}\setlength{\parskip}{0pt}}
\setcounter{secnumdepth}{-\maxdimen} % remove section numbering
\ifLuaTeX
  \usepackage{selnolig}  % disable illegal ligatures
\fi
\usepackage{bookmark}
\IfFileExists{xurl.sty}{\usepackage{xurl}}{} % add URL line breaks if available
\urlstyle{same}
\hypersetup{
  hidelinks,
  pdfcreator={LaTeX via pandoc}}

\author{}
\date{\vspace{-2.5em}}

\begin{document}

library(matlib)

\section{Version 2}\label{version-2}

\subsection{Example 6}\label{example-6}

\subsubsection{The matrix to be solved:}\label{the-matrix-to-be-solved}

A \textless- matrix( c( 0,-2, 2,-1, 1, 2,-2, 1, 2,-1, 0, 1, -2, 0,
1,-1), 4, 4, byrow = TRUE) b \textless- c(-1, 2, 4,-4)

\paragraph{Show (in Latex)}\label{show-in-latex}

showEqn(A, b, fractions = TRUE, latex = TRUE)

\[
\begin{array}{lllllllll}
 0 \cdot x_1 &-& 2 \cdot x_2 &+& 2 \cdot x_3 &-& 1 \cdot x_4  &=&  -1 \\ 
 1 \cdot x_1 &+& 2 \cdot x_2 &-& 2 \cdot x_3 &+& 1 \cdot x_4  &=&   2 \\ 
 2 \cdot x_1 &-& 1 \cdot x_2 &+& 0 \cdot x_3 &+& 1 \cdot x_4  &=&   4 \\ 
-2 \cdot x_1 &+& 0 \cdot x_2 &+& 1 \cdot x_3 &-& 1 \cdot x_4  &=&  -4 \\ 
\end{array}
\]

c(R(A), R(cbind(A,b))) \# show ranks all.equal(R(A), R(cbind(A,b))) \#
consistent?

\begin{Shaded}
\begin{Highlighting}[]
\SpecialCharTok{\textgreater{}} \FunctionTok{c}\NormalTok{(}\FunctionTok{R}\NormalTok{(A), }\FunctionTok{R}\NormalTok{(}\FunctionTok{cbind}\NormalTok{(A,b)))              }\CommentTok{\# show ranks}
\NormalTok{[}\DecValTok{1}\NormalTok{] }\DecValTok{4} \DecValTok{4}
\SpecialCharTok{\textgreater{}} \FunctionTok{all.equal}\NormalTok{(}\FunctionTok{R}\NormalTok{(A), }\FunctionTok{R}\NormalTok{(}\FunctionTok{cbind}\NormalTok{(A,b)))      }\CommentTok{\# consistent?}
\NormalTok{[}\DecValTok{1}\NormalTok{] }\ConstantTok{TRUE}
\end{Highlighting}
\end{Shaded}

\paragraph{Echelon form}\label{echelon-form}

echelon(A, b, verbose=TRUE, fractions=TRUE, reduced = FALSE)

Det(A, verbose = TRUE, fractions = TRUE)

\begin{Shaded}
\begin{Highlighting}[]

\end{Highlighting}
\end{Shaded}

\subsubsection{Ab3}\label{ab3}

Ab3 \textless- matrix( c( 0,-2,-1,-1, 1, 2, 2, 1, 2,-1, 4, 1, -2,
0,-4,-1), 4, 4, byrow = TRUE) b \textless- c(-1, 2, 4,-4)

Det(Ab3, verbose = TRUE, fractions = TRUE)

\begin{Shaded}
\begin{Highlighting}[]
\NormalTok{det }\OtherTok{=}\NormalTok{ (}\SpecialCharTok{{-}}\DecValTok{1}\NormalTok{)}\SpecialCharTok{\^{}}\DecValTok{1}\NormalTok{ x }\DecValTok{2}\NormalTok{ x }\DecValTok{5}\SpecialCharTok{/}\DecValTok{2}\NormalTok{ x }\SpecialCharTok{{-}}\DecValTok{1}\NormalTok{ x }\DecValTok{1}\SpecialCharTok{/}\DecValTok{5} \OtherTok{=} \DecValTok{1}
\end{Highlighting}
\end{Shaded}

\subsubsection{Ab4}\label{ab4}

Ab4 \textless- matrix( c( 0,-2, 2,-1, 1, 2,-2, 2, 2,-1, 0, 4, -2, 0,
1,-4), 4, 4, byrow = TRUE) Det(Ab4, verbose = TRUE, fractions = TRUE)

\begin{Shaded}
\begin{Highlighting}[]
\NormalTok{det }\OtherTok{=}\NormalTok{ (}\SpecialCharTok{{-}}\DecValTok{1}\NormalTok{)}\SpecialCharTok{\^{}}\DecValTok{1}\NormalTok{ x }\DecValTok{2}\NormalTok{ x }\DecValTok{5}\SpecialCharTok{/}\DecValTok{2}\NormalTok{ x }\DecValTok{2}\SpecialCharTok{/}\DecValTok{5}\NormalTok{ x }\DecValTok{1}\SpecialCharTok{/}\DecValTok{2} \OtherTok{=} \SpecialCharTok{{-}}\DecValTok{1}
\end{Highlighting}
\end{Shaded}


\end{document}
