% Options for packages loaded elsewhere
\PassOptionsToPackage{unicode}{hyperref}
\PassOptionsToPackage{hyphens}{url}
%
\documentclass[
  10pt,
  a4paper,
]{article}
\usepackage{amsmath,amssymb}
\usepackage{iftex}
\ifPDFTeX
  \usepackage[T1]{fontenc}
  \usepackage[utf8]{inputenc}
  \usepackage{textcomp} % provide euro and other symbols
\else % if luatex or xetex
  \usepackage{unicode-math} % this also loads fontspec
  \defaultfontfeatures{Scale=MatchLowercase}
  \defaultfontfeatures[\rmfamily]{Ligatures=TeX,Scale=1}
\fi
\usepackage{lmodern}
\ifPDFTeX\else
  % xetex/luatex font selection
\fi
% Use upquote if available, for straight quotes in verbatim environments
\IfFileExists{upquote.sty}{\usepackage{upquote}}{}
\IfFileExists{microtype.sty}{% use microtype if available
  \usepackage[]{microtype}
  \UseMicrotypeSet[protrusion]{basicmath} % disable protrusion for tt fonts
}{}
\makeatletter
\@ifundefined{KOMAClassName}{% if non-KOMA class
  \IfFileExists{parskip.sty}{%
    \usepackage{parskip}
  }{% else
    \setlength{\parindent}{0pt}
    \setlength{\parskip}{6pt plus 2pt minus 1pt}}
}{% if KOMA class
  \KOMAoptions{parskip=half}}
\makeatother
\usepackage{xcolor}
\usepackage{color}
\usepackage{fancyvrb}
\newcommand{\VerbBar}{|}
\newcommand{\VERB}{\Verb[commandchars=\\\{\}]}
\DefineVerbatimEnvironment{Highlighting}{Verbatim}{commandchars=\\\{\}}
% Add ',fontsize=\small' for more characters per line
\usepackage{framed}
\definecolor{shadecolor}{RGB}{248,248,248}
\newenvironment{Shaded}{\begin{snugshade}}{\end{snugshade}}
\newcommand{\AlertTok}[1]{\textcolor[rgb]{0.94,0.16,0.16}{#1}}
\newcommand{\AnnotationTok}[1]{\textcolor[rgb]{0.56,0.35,0.01}{\textbf{\textit{#1}}}}
\newcommand{\AttributeTok}[1]{\textcolor[rgb]{0.13,0.29,0.53}{#1}}
\newcommand{\BaseNTok}[1]{\textcolor[rgb]{0.00,0.00,0.81}{#1}}
\newcommand{\BuiltInTok}[1]{#1}
\newcommand{\CharTok}[1]{\textcolor[rgb]{0.31,0.60,0.02}{#1}}
\newcommand{\CommentTok}[1]{\textcolor[rgb]{0.56,0.35,0.01}{\textit{#1}}}
\newcommand{\CommentVarTok}[1]{\textcolor[rgb]{0.56,0.35,0.01}{\textbf{\textit{#1}}}}
\newcommand{\ConstantTok}[1]{\textcolor[rgb]{0.56,0.35,0.01}{#1}}
\newcommand{\ControlFlowTok}[1]{\textcolor[rgb]{0.13,0.29,0.53}{\textbf{#1}}}
\newcommand{\DataTypeTok}[1]{\textcolor[rgb]{0.13,0.29,0.53}{#1}}
\newcommand{\DecValTok}[1]{\textcolor[rgb]{0.00,0.00,0.81}{#1}}
\newcommand{\DocumentationTok}[1]{\textcolor[rgb]{0.56,0.35,0.01}{\textbf{\textit{#1}}}}
\newcommand{\ErrorTok}[1]{\textcolor[rgb]{0.64,0.00,0.00}{\textbf{#1}}}
\newcommand{\ExtensionTok}[1]{#1}
\newcommand{\FloatTok}[1]{\textcolor[rgb]{0.00,0.00,0.81}{#1}}
\newcommand{\FunctionTok}[1]{\textcolor[rgb]{0.13,0.29,0.53}{\textbf{#1}}}
\newcommand{\ImportTok}[1]{#1}
\newcommand{\InformationTok}[1]{\textcolor[rgb]{0.56,0.35,0.01}{\textbf{\textit{#1}}}}
\newcommand{\KeywordTok}[1]{\textcolor[rgb]{0.13,0.29,0.53}{\textbf{#1}}}
\newcommand{\NormalTok}[1]{#1}
\newcommand{\OperatorTok}[1]{\textcolor[rgb]{0.81,0.36,0.00}{\textbf{#1}}}
\newcommand{\OtherTok}[1]{\textcolor[rgb]{0.56,0.35,0.01}{#1}}
\newcommand{\PreprocessorTok}[1]{\textcolor[rgb]{0.56,0.35,0.01}{\textit{#1}}}
\newcommand{\RegionMarkerTok}[1]{#1}
\newcommand{\SpecialCharTok}[1]{\textcolor[rgb]{0.81,0.36,0.00}{\textbf{#1}}}
\newcommand{\SpecialStringTok}[1]{\textcolor[rgb]{0.31,0.60,0.02}{#1}}
\newcommand{\StringTok}[1]{\textcolor[rgb]{0.31,0.60,0.02}{#1}}
\newcommand{\VariableTok}[1]{\textcolor[rgb]{0.00,0.00,0.00}{#1}}
\newcommand{\VerbatimStringTok}[1]{\textcolor[rgb]{0.31,0.60,0.02}{#1}}
\newcommand{\WarningTok}[1]{\textcolor[rgb]{0.56,0.35,0.01}{\textbf{\textit{#1}}}}
\usepackage{graphicx}
\makeatletter
\def\maxwidth{\ifdim\Gin@nat@width>\linewidth\linewidth\else\Gin@nat@width\fi}
\def\maxheight{\ifdim\Gin@nat@height>\textheight\textheight\else\Gin@nat@height\fi}
\makeatother
% Scale images if necessary, so that they will not overflow the page
% margins by default, and it is still possible to overwrite the defaults
% using explicit options in \includegraphics[width, height, ...]{}
\setkeys{Gin}{width=\maxwidth,height=\maxheight,keepaspectratio}
% Set default figure placement to htbp
\makeatletter
\def\fps@figure{htbp}
\makeatother
\setlength{\emergencystretch}{3em} % prevent overfull lines
\providecommand{\tightlist}{%
  \setlength{\itemsep}{0pt}\setlength{\parskip}{0pt}}
\setcounter{secnumdepth}{-\maxdimen} % remove section numbering
\usepackage[margin=0.75in]{geometry}
\usepackage{fancyhdr}
\let\proof\relax
\let\endproof\relax
\usepackage{amsmath}
\pagestyle{fancy}
\fancyhead{}
\fancyfoot{}
\lhead{LA1 Project}
\rhead{Exercise 2 (Version 2)}
\lfoot{January, 2025}\rfoot{Pg. \thepage}

\ifLuaTeX
  \usepackage{selnolig}  % disable illegal ligatures
\fi
\IfFileExists{bookmark.sty}{\usepackage{bookmark}}{\usepackage{hyperref}}
\IfFileExists{xurl.sty}{\usepackage{xurl}}{} % add URL line breaks if available
\urlstyle{same}
% Make links footnotes instead of hotlinks:
\DeclareRobustCommand{\href}[2]{#2\footnote{\url{#1}}}
\hypersetup{
  pdftitle={LA1 - Version 2 - Exercise 2},
  pdfauthor={Ondrej Salamon},
  hidelinks,
  pdfcreator={LaTeX via pandoc}}

\title{LA1 - Version 2 - Exercise 2}
\author{Ondrej Salamon}
\date{}

\begin{document}
\maketitle

\begin{Shaded}
\begin{Highlighting}[]
\CommentTok{\# initial setup}
\FunctionTok{options}\NormalTok{(}\AttributeTok{scipen =} \DecValTok{999}\NormalTok{)}
\FunctionTok{options}\NormalTok{(}\AttributeTok{tinytex.verbose =} \ConstantTok{TRUE}\NormalTok{)}
\FunctionTok{library}\NormalTok{(matlib)}
\FunctionTok{library}\NormalTok{(knitr)}
\FunctionTok{library}\NormalTok{(rmarkdown)}
\FunctionTok{library}\NormalTok{(quarto)}
\FunctionTok{library}\NormalTok{(tinytex)}
\FunctionTok{library}\NormalTok{(pandoc)}
\NormalTok{knitr}\SpecialCharTok{::}\NormalTok{opts\_chunk}\SpecialCharTok{$}\FunctionTok{set}\NormalTok{(}\AttributeTok{echo=}\ConstantTok{TRUE}\NormalTok{, }\AttributeTok{message=}\ConstantTok{FALSE}\NormalTok{, }\AttributeTok{warning=}\ConstantTok{FALSE}\NormalTok{, }\AttributeTok{fig.width=}\DecValTok{6}\NormalTok{, }\AttributeTok{fig.height=}\DecValTok{6}\NormalTok{)}
\end{Highlighting}
\end{Shaded}

\hypertarget{compute-inverse-of-a}{%
\subsubsection{Compute inverse of A:}\label{compute-inverse-of-a}}

\begin{Shaded}
\begin{Highlighting}[]
\NormalTok{A }\OtherTok{\textless{}{-}} \FunctionTok{matrix}\NormalTok{(}\FunctionTok{c}\NormalTok{(}\DecValTok{2}\NormalTok{,}\SpecialCharTok{{-}}\DecValTok{2}\NormalTok{, }\DecValTok{1}\NormalTok{,}
              \DecValTok{1}\NormalTok{,}\SpecialCharTok{{-}}\DecValTok{1}\NormalTok{, }\DecValTok{1}\NormalTok{,}
             \SpecialCharTok{{-}}\DecValTok{5}\NormalTok{, }\DecValTok{6}\NormalTok{,}\SpecialCharTok{{-}}\DecValTok{4}\NormalTok{), }\DecValTok{3}\NormalTok{, }\DecValTok{3}\NormalTok{, }\AttributeTok{byrow=}\ConstantTok{TRUE}\NormalTok{)}

\NormalTok{Aminus }\OtherTok{\textless{}{-}} \FunctionTok{Inverse}\NormalTok{(A)}
\end{Highlighting}
\end{Shaded}

\hypertarget{code-the-result-in-r}{%
\subsubsection{Code the result in R:}\label{code-the-result-in-r}}

\begin{verbatim}
Eqn(
    "\\mathbf{A^-1} = ",
    latexMatrix(Aminus, matrix="bmatrix")
)
\end{verbatim}

\hypertarget{show-result-in-latex}{%
\subsubsection{Show result (in Latex):}\label{show-result-in-latex}}

\begin{equation*}
\mathbf{A^-1} = \begin{bmatrix} 
 2 &  2 &  1 \\ 
 1 &  3 &  1 \\ 
-1 &  2 &  0 \\ 
\end{bmatrix}
\end{equation*}

\hypertarget{add-other-matrix-b-and-two-vectors-u-and-v}{%
\subsubsection{Add other matrix B and two vectors u and
v:}\label{add-other-matrix-b-and-two-vectors-u-and-v}}

\begin{Shaded}
\begin{Highlighting}[]
\NormalTok{B }\OtherTok{\textless{}{-}} \FunctionTok{matrix}\NormalTok{(}\FunctionTok{c}\NormalTok{( }\DecValTok{2}\NormalTok{, }\DecValTok{0}\NormalTok{, }\DecValTok{2}\NormalTok{,}
              \SpecialCharTok{{-}}\DecValTok{1}\NormalTok{, }\DecValTok{2}\NormalTok{, }\DecValTok{0}\NormalTok{,}
               \DecValTok{3}\NormalTok{, }\DecValTok{3}\NormalTok{, }\DecValTok{2}\NormalTok{), }\DecValTok{3}\NormalTok{, }\DecValTok{3}\NormalTok{, }\AttributeTok{byrow=}\ConstantTok{TRUE}\NormalTok{)}
\end{Highlighting}
\end{Shaded}

\begin{Shaded}
\begin{Highlighting}[]
\NormalTok{u }\OtherTok{\textless{}{-}} \FunctionTok{c}\NormalTok{(}\DecValTok{3}\NormalTok{, }\DecValTok{2}\NormalTok{, }\DecValTok{2}\NormalTok{)}
\end{Highlighting}
\end{Shaded}

\begin{Shaded}
\begin{Highlighting}[]
\NormalTok{v }\OtherTok{\textless{}{-}} \FunctionTok{c}\NormalTok{(}\DecValTok{2}\NormalTok{, }\DecValTok{0}\NormalTok{, }\DecValTok{1}\NormalTok{)}
\end{Highlighting}
\end{Shaded}

\hypertarget{calculate-whole-exercise-alltogether}{%
\subsubsection{Calculate whole exercise
alltogether:}\label{calculate-whole-exercise-alltogether}}

\begin{Shaded}
\begin{Highlighting}[]
\NormalTok{Au }\OtherTok{\textless{}{-}}\NormalTok{ A }\SpecialCharTok{\%*\%}\NormalTok{ u}

\NormalTok{Bv }\OtherTok{\textless{}{-}}\NormalTok{ B }\SpecialCharTok{\%*\%}\NormalTok{ v}

\NormalTok{AuBvplus }\OtherTok{=}\NormalTok{ (Au }\SpecialCharTok{+}\NormalTok{ Bv)}

\NormalTok{final }\OtherTok{=}\NormalTok{ Aminus }\SpecialCharTok{\%*\%}\NormalTok{ AuBvplus}
\end{Highlighting}
\end{Shaded}

\hypertarget{code-exercise-in-matlib-latex-compatible}{%
\subsubsection{Code exercise in matlib +
Latex-compatible:}\label{code-exercise-in-matlib-latex-compatible}}

\begin{verbatim}
Eqn(
    "\\mathbf{A^-1 * (Au + Bv)} =",
    "\\mathbf{A^-1} *",
     latexMatrix(Au, matrix="bmatrix"),
     Eqn_hspace(mid='+'),
     latexMatrix(Bv, matrix="bmatrix"),
     Eqn_hspace(mid='='),
     latexMatrix(Aminus, matrix="bmatrix"),
     Eqn_hspace(mid='*'),
     latexMatrix(AuBvplus, matrix="bmatrix"),
     Eqn_hspace(mid='='),
     latexMatrix(final, matrix="bmatrix")
)
\end{verbatim}

\hypertarget{show-exercise-solution-rs-latex-output}{%
\subsubsection{Show exercise solution (R's Latex
output)}\label{show-exercise-solution-rs-latex-output}}

\begin{equation*}
\mathbf{A^-1 * (Au + Bv)} =\mathbf{A^-1} *\begin{bmatrix} 
  4 \\ 
  3 \\ 
-11 \\ 
\end{bmatrix}
\quad+\quad\begin{bmatrix} 
 6 \\ 
-2 \\ 
 8 \\ 
\end{bmatrix}
\quad=\quad\begin{bmatrix} 
 2 &  2 &  1 \\ 
 1 &  3 &  1 \\ 
-1 &  2 &  0 \\ 
\end{bmatrix}
\quad*\quad\begin{bmatrix} 
10 \\ 
 1 \\ 
-3 \\ 
\end{bmatrix}
\quad=\quad\begin{bmatrix} 
19 \\ 
10 \\ 
-8 \\ 
\end{bmatrix}
\end{equation*}

\end{document}
